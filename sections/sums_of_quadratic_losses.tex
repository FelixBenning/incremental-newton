\section{Sums of Quadratic Losses}

For a family of quadratic losses
\begin{align*}
	\Loss_k(\theta)
	= \Loss_k(w)
	+ \langle\nabla\Loss_k(w), \theta-w\rangle
	+ \tfrac12\langle H_k(\theta-w), \theta-w\rangle
\end{align*}
with \(H_k\) positive definite, we define (with the assumption that the
limit is well defined)
\begin{align}\label{eq: loss limit}
	\Loss(\theta) := \lim_{n\to\infty} \underbrace{\frac1n\sum_{k=1}^n \Loss_k(\theta)}_{=: \Loss_{(n)}(\theta)}.
\end{align}
A finite sum would also work. But this is the more interesting case.
\begin{example}
	for a random loss \(\loss(w, X,Y)\) with
	\begin{align*}
		\Loss(w) := \E[\loss(w, X,Y)],
	\end{align*}
	we can sample independent \((X_k,Y_k)_{k\ge 1}\) according to the distribution of
	\((X,Y)\) and define
	\begin{align*}
		\Loss_k(w) := \loss(w, X_k, Y_k).
	\end{align*}
	We then get \eqref{eq: loss limit} almost surely from the law of large numbers.
\end{example}

\subsection{Tracking Minimas}

Since every \(\Loss_k\) is convex, as \(H_k\) are positive definite, we
simply need to satisfy the first order condition to find a minimum.
\begin{align*}
	0\overset!= \nabla\Loss_k(\theta)
	= \nabla\Loss_k(w) + H_k(\theta-w).
\end{align*}
If \(H_k\) is strictly positive definite, we get for
\begin{align*}
	\minimum_k := w - H_k^{-1}\nabla\Loss_k(w)
\end{align*}
the representation
\begin{align}\label{eq: parabola representation}
	\Loss_k(\theta) = \Loss_k(\minimum_k) + \tfrac12 \|\theta-\minimum_k\|_{H_k}^2
\end{align}
using
\begin{align*}
	\|x\|_H^2 := \langle Hx, x\rangle\ge 0.
\end{align*}
While \(H_k\) strictly positive definite is a sufficient condition for the
representation \eqref{eq: parabola representation}, it is not a necessary
condition.

In particular we get
\begin{align}\label{eq: hessian representation of the gradient}
	\nabla\Loss_k(\theta) = H_k(\theta - \minimum_k)
\end{align}

\begin{example}
	We only need to find \(\minimum_k\) for which \(\nabla\Loss_k(\minimum_k)=0\).
	If we have
	\begin{align*}
		\Loss_k(\theta) = \tfrac12(\langle \theta, X_k\rangle - Y_k)^2
	\end{align*}	
	for example, we need
	\begin{align*}
		0\overset!=\nabla\Loss_k(\theta) = X_k (\langle \theta, X_k\rangle - Y_k)
	\end{align*}
	Here
	\begin{align*}
		\minimum_k = \frac{Y_k}{\|X_k\|^2}X_k
	\end{align*}
	does the job.
\end{example}

\begin{theorem}[Convex Combination of Minima]
	\label{thm: convex combination of minima}
	Assuming we can write \(\Loss_k\) in the form \eqref{eq: parabola representation} and
	\begin{align*}
		H_{(n)}:=\sum_{k=1}^n H_k
	\end{align*}
	is strictly positive definite. Then the minimum of \(\Loss_{(n)}(\theta)\) is a
	generalized convex combination of the minima of \(\Loss_k(\theta)\). More
	specifically
	\begin{align*}
		\minimum_{(n)} = \sum_{k=1}^n [H_{(n)}^{-1} H_k] \minimum_{k}
	\end{align*}
	with
	\begin{align*}
		0\preceq H_{(n)}^{-1}H_k\preceq \identity
		\quad \text{and}\quad
		\sum_{k=1}^n [H_{(n)}^{-1} H_k] = \identity
	\end{align*}
	where \(A\preceq B\) is defined as ``\(B-A\) is positive definite''. Here
	this also means that all eigenvalues of \(H_{(n)}^{-1}H_k\) are between
	\(0\) and \(1\).
\end{theorem}
\begin{proof}
	We have \(\Loss_{(n)}\) is strictly convex, so we need
	\begin{align*}
		0 \overset!= \nabla\Loss_{(n)}(\theta)
		=\frac1n \sum_{k=1}^n \nabla\Loss_k(\theta)
		= \frac1n \sum_{k=1}^n H_k(\theta - \minimum_k)
	\end{align*}
	By linearity it follows that
	\begin{align*}
		H_{(n)} \minimum_{(n)} = \sum_{k=1}^n H_k \minimum_k,
	\end{align*}
	which implies the claim.
\end{proof}

\begin{corollary}
	In the same setting as Theorem~\ref{thm: convex combination of minima}, we
	can also define
	\begin{align*}
		\bar{H}_{(n)} := \frac1n\sum_{k=1}^n H_k = \nabla^2\Loss_{(n)} \to \nabla^2\Loss
	\end{align*}
	which converges if \(\nabla^2\Loss\) exists, and due to \(H_{(n)}^{-1}H_k =
	\tfrac1n\bar{H}_{(n)}^{-1}H_k\) we get
	\begin{align*}
		\minimum_{(n)}
		= \frac1n\sum_{k=1}^n [\bar{H}_{(n)}^{-1} H_k] \minimum_{k}.
	\end{align*}
\end{corollary}
